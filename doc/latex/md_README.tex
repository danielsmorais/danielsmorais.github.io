Universidade Federal do Rio Grande do Norte -\/ Departamento de Engenharia de Computação e Automação -\/ D\+CA \section*{Time Poti de Futebol de Robôs }

\begin{quote}
Para utilização do sistema do Time Poti é necessário a utilização de um sistema operacional Linux. \end{quote}


\begin{quote}
A manipulação do sistema é feita via Terminal do Linux, tendo a interface gráfica apenas o módulo de calibração. \end{quote}


\subsection*{Getting Started }

\subsubsection*{Instalação de bibliotecas e pacotes}

Inicialmente, é interessante que seu sistemas esteja atualizado. Para isso, deve ser utilizado o comando\+:


\begin{DoxyCode}
1 sudo apt-get update
\end{DoxyCode}


Comando para instalação de pacotes para utilização do bluetooth 
\begin{DoxyCode}
1 sudo apt-get install bluetooth bluez-utils blueman
\end{DoxyCode}


Após esse comado, verificar se na pasta include do computador está a pasta {\ttfamily /usr/include/bluetooth}. Caso não, use o comando a seguir\+: 
\begin{DoxyCode}
1 sudo apt-get install build-essential libbluetooth-dev libdbus-1-dev
\end{DoxyCode}


Deve ser instalada a versão 4 do Qt, logo, basta utilizar o comando a seguir\+: 
\begin{DoxyCode}
1 sudo apt-get install libqt4-dev qt4-dev-tools
\end{DoxyCode}


\subsubsection*{Make Make Make Make Make...}

Agora, utilizar os comandos a seguir em todas as pastas da pasta {\ttfamily libsrc}\+: 
\begin{DoxyCode}
1 make
2 make clean
3 make install
\end{DoxyCode}


Ir nas pastas {\ttfamily program/calibrador} e {\ttfamily program/main} e utilizar os comandos\+: 
\begin{DoxyCode}
1 make clean
2 make 
\end{DoxyCode}


Após essas etapas, você pode fazer a {\ttfamily calibração} do campo com seus objetos ou fazer um jogo. Caso opte em calibração, deve-\/se verificar se a câmera a ser usada está referenciada corretamente. ``` \subsubsection*{Calibração e Jogo}

Para fazer a calibração, basta executar em {\ttfamily program/calibrador} 
\begin{DoxyCode}
1 ./calibrador
\end{DoxyCode}


Ou

Para fazer um jogo, basta executar em {\ttfamily program/main} 
\begin{DoxyCode}
1 ./main
\end{DoxyCode}
 \begin{quote}
Divirta-\/se!!\end{quote}
